\documentclass[calculus]{subfiles}

\begin{document}
    \section{Понятие множества. Понятие отображения.}\label{sec:part-1}

    \begin{definition}[Множество]
        Множеством называют неупорядоченное объединение вполне различимых объектов.
    \end{definition}

    \bigskip

    Способы задания множества:
    \begin{enumerate}
        \item Перечисление \(\{1, 2\}\)
        \item Задание свойства \(\{x : P(x)\}\)
        \item Множество множеств \(\{\Lambda_i\}_{i \in I}\)
    \end{enumerate}

    \bigskip

    Операции над множествами:
    \begin{enumerate}
        \item \((A \subset B) := \forall x \in A : x \in B \)
        \item \((A = B) := A \subset B \land B \subset A\)
        \item \(A \cup B\) - объединение
        \item \(A \cap B\) - пересечение
        \item \(A \setminus B\) - разность
        \item \(\neg A\) - дополнение
        \item \(A \  \triangle \  B\ = (A \setminus B) \cup (B \setminus A) \) - симметрическая разность
        \item \(A \times B\) - прямое произведение
    \end{enumerate}

    \bigskip

    Законы де Моргана:
    \begin{enumerate}
        \item \(A \setminus (B \cup C) = (A \setminus B) \cap (A \setminus C)\)
        \item \(A \setminus (B \cap C) = (A \setminus B) \cup (A \setminus C)\)
    \end{enumerate}

    \bigskip

    Дестрибутивность:
    \begin{enumerate}
        \item \(A \cup (B \cap C) = (A \cup B) \cap (A \cup C) \)
        \item \(A \cap (B \cup C) = (A \cap B) \cup (A \cap C) \)
    \end{enumerate}

    \begin{n_definition}[Функция]
        Функцией $f$ от $A$ в $B$ называют сопоставление $\forall x \in A \to f(x) \in B$
    \end{n_definition}

    \begin{notation}
        Для функции $f$ из $A$ в $B$:
        \begin{itemize}
            \item $A$ - область определения
            \item $B$ - область прибытия
            \item $f(A)$ - область значений
        \end{itemize}
    \end{notation}

    \begin{definition}[График функции]
        $Г_f = \{(x, f(x)) : x \in A\}$
    \end{definition}

    \begin{definition}[Функция]
        Функцией из $A$ в $B$ называют упорядоченную тройку $(A, B, Г)$, где A - область определения, B - область прибытия, Г - график функции.
    \end{definition}

    \begin{notation}[Функция]
        $f: A \to B$ - функция из $A$ в $B$
    \end{notation}

    \begin{notation}[Сужение функции]
        $\forall C \subset A:$ сужением $f$ на $C$ называется $f|_C$ т.ч.  $f|_C(x) = f(x)$  и $Г_{f|_C}$ = $Г_f \cap (С \times B) $
    \end{notation}

    \begin{notation}[Образ множества]
        $\forall X \subset A: f(X) = \{f(x) : x \in X\}$ - образ множества X
    \end{notation}

    \begin{notation}[Прообраз множества]
        $\forall Y \subset B: f^{-1}(Y) = \{x : f(x) \in Y\}$ - прообраз множества Y
    \end{notation}

    Отображение $f: A \to B$ называют:
    \begin{itemize}
        \item Инъективным, если $\forall x_1, x_2 \in X: (f(x_1) = f(x_2)) \implies (x_1 = x_2)$
        \item Сюръективным, если $f(A) = B$
        \item Биективным (или взаимно однозначным), если оно сюръективно и инъективно одновременно
    \end{itemize}

    \begin{notation}[Композиция функций]
        $\forall (f:A \to B, \  g: B \to C): \exists ((g \circ f) : A \to C )$ - композиция $f$ и $g$ т.ч. $\forall x \in A: (g \circ f)(x) = g(f(x))$
    \end{notation}

    \begin{remark}
        Композиция ассоциативна
    \end{remark}

    \begin{proof}
        $((h \circ g) \circ f)(x) = (h \circ g)(f(x)) = h(g(f(x))) = h((g \circ f)(x)) = (h \circ (g \circ f))(x)$
    \end{proof}

    \begin{definition}[Тождественное отображение]
        $id_A: A \to A$ т.ч. $id(x)=x$ - тождественное отображение
    \end{definition}

    \begin{definition}[Обратный]
        $g: B \to A$ называется правым (левым) обратным к $f: A \to B$, если $f \circ g = id_B$ ($g \circ f = id_A$)
    \end{definition}

    \begin{remark}
        Если у отображения есть правый и левый обратные, то они равны.
    \end{remark}

    \begin{remark}
        .
        \begin{enumerate}
            \item f инъективно $\iff$ f обратимо слева
            \item f сюръективно $\iff$ f обратимо справа
            \item f биективно $\iff$ f обратимо
        \end{enumerate}
    \end{remark}

    \begin{proof}
        $\forall (f: A \to B)$:\\
        1.  $\bimplies$:\\
        Обозначим $g$ - левое обратное к $f$ \\
        $\forall x_1, x_2$ т.ч. $f(x_1) = f(x_2)$: $x_1 = g(f(x_1)) = g(f(x_2)) = x_2$ \\
        1.  $\implies$:\\
        Выберем $z \in  A$ \\
        Рассмотрим $g(y) = x$, если $\exists x$ т.ч $f(x)=y$. В противном случае $g(y) = z$ \\
        $g(f(x)) = x$ (по построению) $\implies$ $g$ - обратное слево.\\
        \\
        2.  $\bimplies$: \\
        Обозначим $g$ - правое обратное к $f$ \\
        $\forall y \in B: \exists g(y)$ т.ч. $f(g(y)) = y$\\
        2.  $\implies$: \\
        $\forall y \in B: \exists x \in A$ т.ч. $f(x) = y$ \\
        Пусть $g(y) = x \implies f(g(y)) = y \implies g$ - обратное справа.
    \end{proof}
\end{document}
