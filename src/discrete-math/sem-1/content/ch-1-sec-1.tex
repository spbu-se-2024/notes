\section{Множества}\label{sec:ch-1-sec-1}


\begin{definition}[Наивная теория множеств]
    \begin{compactenum}
        \item Множество однозначно определяется своими объектами и может быть задано простым перечислением. $A = \set{1, \lambda, word, \set{IV, white}}$
        \item Порядок элементов в множестве не имеет значения. $\set{a, b} = \set{b, a}$
        \item Будем говорить, что элемент $a$ принадлежит множеству $A$, или что этот элемент содержится в нём если $A = \set{a, \ldots}$, и будем обозначать это $a \in A$.
        \item Множество может быть задано с помощью предиката. $A = \set{x : P(x)}$ или $A = \set{x \mid P(x)}$ (последнее будет использоваться чаще)
    \end{compactenum}
\end{definition}

\begin{definition}[Подмножество]
    Будем говорить, что множество $A$ является \definiendum{подмножеством} множества $B$,
    или что $B$ является \definiendum{надмножеством} $A$ если каждый элемент в $A$ также содержится в $B$.
    \\[1ex]
    Т.е. $A \subseteq B \coloneqq \forall x \in A \imp x \in B$; $\subseteq$ --- \definiendum{отношение вложения}.
\end{definition}

\begin{sh-proposition}
    $A \subseteq B \land B \subseteq A \iff A = B$
\end{sh-proposition}

\begin{definition}[Собственное подмножество]
    $\subset$ --- \definiendum{отношение строгого вложения} (аналогично \definiendum{собственное подмножество}):
    \\[1ex]
    $A \subset B \coloneqq A \subseteq B \land A \neq B$
\end{definition}

\begin{examples}[Множества и теоретико-множественные записи]
    \begin{compactenum}
        \item $\Phi = \set{1, \lambda, \textrm{слово}, \set{IV, white}}$
        \item $\textrm{Щ} = \set{1, a} \quad \textrm{Ъ} = \set{a, 1}$
        \item $\textrm{слово} \in \Phi \quad \set{\lambda, \textrm{слово}} \subseteq \Phi$
        \item $\set{IV, white} \not\subseteq \Phi$
        \item $\Lambda = \set{\set{IV, white}} \subseteq \Phi$
        \item $\Lambda \subset \Phi$
    \end{compactenum}
\end{examples}

\begin{definition}[Простые опреации над множествами]
    \begin{compactitem}
        \item \definiendum{Объединение множеств}: $A \cup B = \set{x \mid x \in A \lor x \in B}$
        \item \definiendum{Пересечение множеств}: $A \cap B = \set{x \mid x \in A \land x \in B}$
        \item \definiendum{Разность множеств}: $A \setminus B = \set{x \mid x \in A \land x \notin B}$
        \item \definiendum{Симметрическая разность множеств}: $A \symdif B = \p{A \setminus B} \cup \p{B \setminus A}$
    \end{compactitem}
\end{definition}

\begin{sh-designation}[Пустое множество]
    $\varnothing = \set{}$
\end{sh-designation}

\begin{sh-definition}[Множество натуральных чисел]
    $\N = \set{1, 2, 3, \ldots}$
\end{sh-definition}

\begin{sh-designation}[Отрезок натурального ряда]
    $m:n \coloneqq
    \begin{cases}
        \set{m, m + 1, \ldots, n}, & m < n\\
        \set{m}, & m = n\\
        \varnothing, & m > n
    \end{cases}$
\end{sh-designation}

\begin{designation}[Множество множеств]
    $\Lambda = \set{\Lambda_i \mid \forall i \in I}$, где $I$ --- множество индексов, в данном контексте обычно дискретное (то есть конечное или счётное).
\end{designation}

\begin{sh-designation}[Объединение множества множеств]
    $\bigcup \Lambda = \set{x \mid \exists i \in I\ x \in \Lambda_i}$
\end{sh-designation}

\begin{sh-designation}[Пересечение множества множеств]
    $\bigcap \Lambda = \set{x \mid \forall i \in I\ x \in \Lambda_i}$
\end{sh-designation}

\begin{sh-definition}[Булеан]
    \label{Булеан}
    $\2{A}$ --- множество всех подмножеств множества $A$.
    Также называют \definiendum{булеаном}.
\end{sh-definition}

\begin{sh-remark}
    Не надо читать это~\thref{Булеан} как <<два в степени A>>, по крайней мере, на экзамене у Абрамовской так делать точно не стоит.
\end{sh-remark}

\begin{sh-example}
    $\Phi$ задано выше.
    Тогда $\set{1, \lambda} \subseteq \Phi \imp \set{1, \lambda} \in \2{\Phi}$
\end{sh-example}

\begin{sh-remark}
    $\forall A \imp \varnothing \subseteq A, \varnothing \in \2{A}, \varnothing \subseteq \2{A}$
\end{sh-remark}

\begin{sh-definition}[Мощность множества]
    $\abs{A}$ --- число элементов в множестве $A$ или \definiendum{мощность} $A$
\end{sh-definition}

\begin{sh-designation}[Конечное множество]
    $A$ --- конечно $\iffdef \abs{A} < \infty$
\end{sh-designation}

\begin{sh-proposition}[Мощность булеана]
    $\forall A : \abs{A} < \infty \imp \abs{\2{A}} = 2^{\abs{A}}$
\end{sh-proposition}

\begin{proof}
    Для доказательства данного факта воспользуемся методом математической индукции.

    База индукции: $A = \varnothing \imp \abs{A} = 0, \2{A} = \set{\varnothing} \imp \abs{\2{A}} = 1$.

    Переход индукции: пусть $\abs{X} = n + 1$ и известно, что $\forall A : \abs{A} = n \imp \abs{\2{A}} = 2^n$, покажем, что $\abs{\2{X}} = 2^{n + 1}$.
    Поскольку $n \geq 0 \imp \exists x \in X$, рассмотрим $X \setminus \set{x} = \widetilde{X} : \abs{\widetilde{X}} = n \imp \abs{\2{{\widetilde{X}}}} = 2^n$.\\
    $\2{X}=\2{{\widetilde{X}}} \cup \set{\chi\cup\set{x} \mid \chi \in \2{{\widetilde{X}}}}$

    Поскольку $\2{{\widetilde{X}}} \cap \set{\chi\cup\set{x} \mid \chi \in \2{{\widetilde{X}}}} = \varnothing \imp \abs{\2{X}} = \abs{\2{{\widetilde{X}}}} + \abs{\set{\chi\cup\set{x} \mid \chi \in \2{{\widetilde{X}}}}} = 2^n + 2^n = 2^{n + 1}$.
\end{proof}