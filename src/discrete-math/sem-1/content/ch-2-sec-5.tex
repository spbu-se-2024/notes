\section{Топологическая сортировка}\label{sec:ch-2-sec-5}

В этом параграфе под фразой \guillemotleftчастичный порядок\guillemotright ~будет подразумеваться \guillemotleftчастичный порядок или строгий частичный порядок\guillemotright.

\begin{lemma}[О сужении бинарного отношения]
    Пусть $R$ — рефлексивное / антирефлексивное / симметричное / антисимметричное / асимметричное / транзитивное / антитранзитивное отношение на $A$, тогда $\forall ~X \subseteq A ~R(X)=R\cap X^2$ (a.k.a. \textbf{сужение} $R$ на $X$) — тоже.
\end{lemma}

\begin{sh-proof}
    Тривиально и предоставляется читателю в качестве лёгкого упражнения.
\end{sh-proof}

\begin{lemma}[О существовании минимального элемента]
    Во всяком конечном частично упорядоченном непустом множестве существует минимальный элемент.
\end{lemma}

\begin{proof}
    Кратко: индукция по мощности множества.

    Подробно: для доказательства данной леммы воспользуемся методом математической индукции.

    \textit{База индукции:} $X = \{x\}$. Минимальный элемент очевидно существует и равен $x$.

    \textit{Переход индукции:} пусть $|X| = n + 1, n \geq 1$ и известно, что в любом множестве мощностью $n$ существует минимальный элемент. Тогда выберем $x \in X$ и рассмотрим $X' = X \backslash \{x\}$. По предположению индукции в $X'$ есть минимальный элемент, обозначим его $x_0$. Если $(x, x_0) \in R$, то $x$ — минимальный элемент $X$, так как $\forall ~x' \in X (x', x) \notin R$ (в противном случае $(x', x_0) \in R$, из чего следует, что $x_0$ — не минимальный элемент $X'$). Иначе $x_0$ остаётся минимальным элементом.
\end{proof}

\begin{sh-definition} [Топологическая сортировка]
    poset $A$ отн. $R$ — такой лин. порядок $Q$, что $R \subseteq Q$.
\end{sh-definition}

\begin{theorem}
    У любого конечного частично упорядоченного относительно $R$ множества $A$ существует топологическая сортировка.
\end{theorem}

\begin{proof}
    Конструктивное док-во (чтобы доказать существование чего-нибудь, предъявим нечто и покажем, что это именно то, что мы ищем). Рассмотрим частный случай: $A = \varnothing$. Очевидно, что тогда топологическая сортировка существует и равна $\varnothing$. Теперь предположим, что $A \neq \varnothing$. Для начала упорядочим элементы $A$. Для этого воспользуемся методом математической индукции, но не совсем. Поскольку $A$ непустое, в нём (по лемме 2) существует минимальный элемент. Обозначим его за $m_0$, а само множество $A$ за $A_0$.

    Пусть $A_i \backslash \{m_i\} \neq \varnothing$. Тогда $A_{i + 1} = A_i \backslash \{m_i\}$, $m_{i+1} = min A_i$. Заметим, что $A_{i+1} \subset A_i \Rightarrow A_j \subset A_i$ при $i < j$.

    Обозначим $Q = \{(m_i, m_j) : i \leq j\}$, если $R$ — частичный порядок, и $Q = \{(m_i, m_j) : i < j\}$, если $R$ — строгий частичный порядок. Покажем, что $Q$ — искомая топологическая сортировка.

    Очевидно, что рефлексивность / антирефлексивность следует из построения $Q$. Антисимметричность тоже. Транзитивность тривиальна. Линейность тоже. Следовательно, $Q$ — отношение линейного порядка на $A$. Покажем, что $R \subseteq Q$.

    Предположим противное: существуют $i > j$ такие, что $(m_i, m_j) \in R$. Тогда $m_i \notin A_j$, что противоречит тому, что $A_i \subset A_j$, что следует из построения $A_i$ и $m_i$. Соответственно, $R \subseteq Q$ и всё хорошо.
\end{proof}

\begin{sh-remark}
    $R^{-1} = \{(y, x) : (x, y) \in R\}$ — частичный порядок, если $R$ — частичный порядок $\Rightarrow min R = max R^{-1}$.
\end{sh-remark}

\begin{sh-remark}
    $A$ — строго poset относительно $R \Rightarrow Q$ — строгий линейный порядок.
\end{sh-remark}